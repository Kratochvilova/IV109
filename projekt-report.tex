\documentclass[10pt,a4paper]{report}
\usepackage[utf8]{inputenc}
\usepackage[czech]{babel}
\usepackage[T1]{fontenc}
\usepackage{amsmath}
\usepackage{amsfonts}
\usepackage{amssymb}
\author{Pavla Kratochvílová, Adriana Šmijáková}
\title{IV109 - Formace názorů}

\begin{document}
\maketitle
% stručné uvedení do tématu, objasnění základních pojmů
\chapter{Úvod}

\chapter{Návrh modelu}
% přesná formulace modelovaného problému, případná relevantní data
\section{Formulace problému}
Naším cílem je zachytit proces formování názorů, který probíhá při interakcích lidí v sociální síti. Vycházíme z myšlenky, že názory člověka jsou do značné míry ovlivněny názory v jeho okolí. Uvažujeme přitom spojitý názor -- člověk nemusí pouze zcela přijmout nebo zamítnout názor svého okolí, ale může jen lehce poupravit svůj vlastní názor tak, aby se více přiblížil názoru okolí. 

Formování názorů zkoumáme na několika různých typech sítí a s různým počátečním rozložením názorů. 

% popis zvoleného přístupu k modelování a základních prvků modelu, vztahů a zpětných vazeb, vysvětlení základních rovnic/pravidel
\section{Model}
Formování názorů modelujeme pomocí agentů, kdy každý agent má přidělen názor vyjádřený reálným číslem z intervalu $\langle 0, 1 \rangle$. Agenti jsou vzájemně propojeni v rámci určitého grafu a interagují spolu pouze agenti spojení vazbou.

Model se skládá ze dvou částí: nejprve se vytvoří síť agentů a každému se nastaví názor, a poté probíhá samotný proces formování názorů.

\subsection{Tvorba grafu}
Všechny grafy jsou parametrizovány počtem vrcholů (parametr \texttt{people}) a průměrným stupněm vrcholů (parametr \texttt{average-node-degree}). Uvažovali jsme čtyři různé typy grafů:

\begin{itemize}
	\item Náhodný graf vzniká náhodným přidáváním hran, dokud neodpovídá průměrný stupeň vrcholů.
	\item Prostorový graf (spatial graph) je převzatý z již existujícího modelu \textit{Virus on a Network}. Je vytvořen náhodným rozmístěním agentů do prostoru a následným přidáváním hran, dokud neodpovídá průměrný stupeň vrcholů. Hrany jsou přidávány mezi náhodným agentem a jemu nejbližším agentem, se kterým ještě není spojen.
	\item Malý svět je založený na existujícím modelu \textit{Small World}. Nejprve je vytvořen cyklus všech agentů, a poté jsou přidány hrany mezi agenty vzájemně vzdálenými méně než nějaké n tak, aby se stupeň vrcholu co nejvíce přiblížil zvolenému průměrnému stupni. Následně některé hrany zamění jeden svůj konec za náhodný jiný. Počet takových hran je dán nepřímo parametrem \texttt{rewiring-prob}.
	\item Preferenční graf je založený na existujícím modelu \textit{Prefferential Attachment}. Vzniká postupným přidáváním vrcholů, přičemž nový vrchol se spojí s nějakým již existujícím. Upřednostňovány jsou vrcholy, které mají více hran -- toho je dosaženo náhodným výběrem z konců hran raději než výběrem z vrcholů. V preferenčním grafu se parametr \texttt{average-node-degree} nebere v úvahu.
\end{itemize}

Výběr typu grafu lze provést pomocí parametru \texttt{network-type}.

\subsection{Počáteční rozložení názorů}
Agentům jsou přiřazeny názory v intervalu $\langle 0, 1 \rangle$. Uvažovali jsme tři možná počáteční rozložení názorů:

\begin{itemize}
	\item Uniformní rozložení.
	\item Normální rozložení -- s průměrem $0.5$ a standardní odchylkou $0.2$.
	\item Normální rozložení se středem v extrémech -- stejné jako normální rozložení v předchozím bodě, ale posunuté tak, aby nejvíce názorů bylo extrémních (tj. těsně nad $0$ a těsně pod $1$) a nejméně středových (tj. okolo $0.5$).
\end{itemize}

Počáteční rozložení názorů lze změnit pomocí parametru \texttt{opinion-distribution}. 

\subsection{Změny názorů}
Uvažujeme dvě strategie pro změnu názorů:

\begin{itemize}
	\item Jeden soused -- agent se podívá na jednoho náhodného agenta ze svého okolí a podle jeho názoru změní svůj vlastní názor.
	\item Všichni sousedé -- agent se podívá na všechny agenty ve svém okolí a podle jejich průměrného názoru změní svůj vlastní názor.
\end{itemize}

V obou případech proběhne změna názoru stejným způsobem. Agent nejprve vypočte rozdíl mezi svým názor a názorem okolí (tj. buď názor vybraného souseda, nebo průměrný názor všech sousedů), a následně posune svůj názor o zlomek tohoto rozdílu směrem k názoru okolí. Velikost zlomku je daná parametrem \texttt{changing-opinion-strength}.

\chapter{Analýza a interpretace výsledků}
V tejto kapitole sa pozrieme na vplyv rôznych parametrov na priebeh formovania názorov. Z návrhu modelu vyplýva, že pokiaľ je graf vzťahov medzi ľuďmi spojitý, stav modelu sa vždy blíži k určitej dohode medzi agentmi (tzn. agenti sa dohodnú na nejakom jednotnom názore - prípadne na názore s malými odchýlkami). My pozorujeme rýchlosť akou sa agenti dohodnú, prípadne distribúciu názorov v krátkodobom horizonte, keď agenti preberajú názory iných len obmedzený čas.

\section{Sila zmeny názoru}
Parameter silu zmeny názoru skúmamame na každom type grafu. Fixujeme pritom parametre: stratégiu preberania názorov (na \i{one neighbor}), počiatočnú distribúciu (na \i{uniform}) a počet ľudí (100).  Každý obrázok znázorňuje porovnanie štyroch rôznych hodnôť sily (0.1, 0.5, 0.7, 1) pre daný typ grafu. Každá kombinácia pohyblivých parametrov 
\chapter{Závěr}

\end{document}